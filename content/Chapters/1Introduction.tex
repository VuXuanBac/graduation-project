\documentclass[../main.tex]{subfiles}
\begin{document}\label{chap1}

Chương đầu tiên của đồ án bắt đầu với việc nêu lên vấn đề cần giải quyết, đó là bài toán thiết kế an toàn bảo mật trong mạng vô tuyến nhận thức. Vấn đề nêu lên không phải là mới và thực tế có rất nhiều tác giả quan tâm nghiên cứu. Một số nghiên cứu cũng được giới thiệu trong chương này và được trình bày theo các hướng tiếp cận khác nhau. Trên cơ sở xác định những hạn chế của các nghiên cứu trước đây, đồ án lựa chọn mô hình cho vấn đề và nêu lên mục tiêu của nghiên cứu. Kết quả nghiên cứu được tóm tắt và tổng kết thành những đóng góp chính. Cuối chương là bố cục của đồ án, trình bày tổng quan nội dung của các chương tiếp theo.

\section{Đặt vấn đề}
\label{sec:dvd}

Mạng vô tuyến nhận thức (cognitive radio network - CRN) là một công nghệ hứa hẹn giúp tăng hiệu quả sử dụng phổ tần số \cite{liang2011}, ở đó, bên thứ cấp (secondary user - SU) vẫn có cơ hội được truyền tin trên dải tần số vốn chỉ dành cho bên sơ cấp (primary user - PU) sử dụng. Cũng giống như các mạng truyền thông khác, CRN cũng là mục tiêu tấn công an toàn bảo mật, đặc biệt là tấn công nghe lén, vì trong hệ thống có nhiều người dùng truyền tin. Một hướng tiếp cận an toàn bảo mật, mặc dù đã có nền tảng lý thuyết từ rất sớm, nhưng mới được quan tâm nghiên cứu gần đây là bảo mật lớp vật lý (physical layer security - PLS). Với các kết quả nghiên cứu sâu rộng, PLS cho thấy một giải pháp giúp tăng cường an toàn bảo mật trong truyền thông không dây, bao gồm cả CRN. Tuy nhiên, thiết kế an toàn bảo mật trong CRN có phần thách thức hơn, đặc biệt khi xem xét bên thứ cấp đồng thời truyền tin với bên sơ cấp, vì khi đó yêu cầu an toàn cần được đảm bảo ở cả hai bên. Ngoài ra, trong quá trình truyền tin, tín hiệu từ cả hai bên có thể tác động lẫn nhau, từ đó cũng đặt ra thêm yêu cầu về chất lượng truyền tin tin cậy và truyền tin an toàn cho bài toán thiết kế.

\section{Các giải pháp hiện tại và hạn chế}
\label{sec:giaiphap}

Từ góc nhìn thiết kế hệ thống, các nghiên cứu về đảm bảo an toàn bảo mật trong mạng vô tuyến nhận thức thường tập trung vào các chủ đề (i) xử lý tín hiệu, (ii) lựa chọn hợp tác và (iii) cấp phát tài nguyên \cite{wang2018survey}. Các kỹ thuật liên quan đến xử lý tín hiệu như định hướng chùm tia (beamforming) hay sử dụng nhiễu nhân tạo (artificial noise - AN) thường được quan tâm nghiên cứu. Dựa trên kỹ thuật định hướng chùm tia, tác giả Kwon \cite{kwon2012secure} đề xuất chiến lược cực đại hóa tốc độ truyền tin an toàn cho PU trong mạng CRN nhiều ăng-ten phát một ăng-ten thu (multiple input single output - MISO) với các mức độ khác nhau về thông tin mà bên phát có về kênh truyền nghe lén. Với mục tiêu thiết kế là cực đại hiệu quả năng lượng an toàn (secrecy energy efficiency - SEE) của hệ thống, tác giả Ouyang \cite{ouyang2017secrecy} đề xuất chiến lược định hướng chùm tia tối ưu (optimal beamforming) và định hướng chùm tia dựa trên kỹ thuật ép không (zero-forcing based beamforming) để giải quyết bài toán tối ưu. Tác giả He \cite{he2014secrecy} đã đề xuất chiến lược truy cập phổ cho các bên của CRN theo hai pha. Ở đó, trong pha truyền tin của mình, SU cần dành một phần năng lượng để phát lại tín hiệu tới bên nhận PU, đồng thời sinh nhiễu nhân tạo để cản trở kẻ nghe lén giải mã các thông điệp từ cả hai bên. Nhìn chung, kỹ thuật định hướng chùm tia được sử dụng nhằm khai thác khả năng tập trung năng lượng, từ đó giúp gia tăng chất lượng tín hiệu tại bên nhận hợp pháp. Nhiễu nhân tạo lại được thiết kế nhằm làm suy giảm chất lượng tín hiệu nhận tại bên nghe lén.

Hợp tác và xếp lịch cho các bên cũng là một hướng tiếp cận hiệu quả cho bài toán an toàn bảo mật. Với mô hình nhiều PU, nhiều SU và nhiều kẻ nghe lén, tác giả Houjeij và cộng sự \cite{houjeij2013game} đã phân tích tương tác giữa các SU và kẻ nghe lén theo lý thuyết trò chơi (game theory). Trong mô hình đó, mỗi SU cần chọn kênh truyền của một trong các PU để truyền tin an toàn và hạn chế ảnh hưởng can nhiễu từ các SU khác; đồng thời, mỗi kẻ tấn công cần chọn một số kênh truyền của PU để nghe lén, đảm bảo tốc độ truyền tin an toàn của các SU là nhỏ nhất. Ở hướng tiếp cận khác, khi xem xét mô hình có nhiều SU cùng truyền tin tới một trạm gốc (base station) và nhiều kẻ nghe lén hợp tác với nhau, tác giả Zou \cite{zou2014secrecy} đề xuất một chiến lược lập lịch theo giải thuật định thời luân phiên (round-robin) cho các SU, đảm bảo hiệu năng an toàn của hệ thống là cao nhất. Tác giả Quach và các cộng sự \cite{quach2017secrecy} đề xuất một chiến lược tự điều chỉnh công suất phát cho các bên kết hợp với việc lựa chọn bộ phát lại (relay). Kết quả cho thấy tính hiệu quả của chiến lược hợp tác trong việc đảm bảo an toàn cho SU. Cũng dựa trên sự hỗ trợ của các bộ phát lại, tác giả Bouabdellah \cite{bouabdellah2018secrecy} xem xét trường hợp sử dụng bộ phát lại với nhiều ăng-ten thu và một ăng-ten phát, nhằm tăng cường tín hiệu cho SU. Kết quả cho thấy việc gia tăng số lượng ăng-ten thu cho bộ phát lại giúp tăng cường hiệu năng an toàn cho hệ thống. Có thể thấy, khi mở rộng mô hình với nhiều người dùng, ở đó có sự đa dạng về vị trí của các đối tượng, một chiến lược lựa chọn hợp tác phù hợp cũng giúp tăng cường hiệu năng an toàn cho toàn hệ thống.

Hướng tiếp cận an toàn bảo mật theo cấp phát tài nguyên thường liên quan đến việc sử dụng hiệu quả các tài nguyên như tần số, khe thời gian và năng lượng. Tác giả Wu \cite{wu2011information} đã mô hình hóa bài toán xác định công suất phát cho các bên nhằm tối ưu tốc độ truyền an toàn ở PU và tốc độ truyền tin tin cậy ở SU theo lý thuyết trò chơi. Trong đó, tác giả đã xem xét lợi ích của PU trong vấn đề thiết kế khi PU có quyền ưu tiên chọn trước công suất phát và có thể quyết định không cùng truyền với SU nếu hiệu quả an toàn không cải thiện hơn khi truyền đơn lẻ. Khi nghiên cứu hiệu năng của CRN với một ăng-ten phát nhiều ăng-ten thu (single input multiple output - SIMO), tác giả Hung \cite{tran2017cognitive} đề xuất chiến lược lựa chọn công suất phát cho SU đảm bảo truyền tin tin cậy và an toàn cho PU, đồng thời xem xét ảnh hưởng công suất phát từ PU lên chất lượng truyền tin an toàn của SU. 

Như vậy, các nghiên cứu an toàn bảo mật trong mạng vô tuyến nhận thức được quan tâm nghiên cứu và khai thác ở nhiều khía cạnh khác nhau, cho thấy nhiều hướng tiếp cận tiềm năng giúp giải quyết vấn đề này. Tuy nhiên, các nghiên cứu này có một số giới hạn. Thứ nhất, bài toán thiết kế chỉ xem xét tính an toàn cho PU hoặc cho SU \cite{bouabdellah2018secrecy,quach2017secrecy,houjeij2013game,kwon2012secure,wu2011information,zou2014secrecy,ouyang2017secrecy}. Với nghiên cứu \cite{tran2017cognitive}, tác giả xem xét tính an toàn cho cả hai bên nhưng thiết kế công suất phát chỉ được đề cập cho bên SU. Thứ hai, để có được các chiến lược thiết kế có hiệu năng tốt, các nghiên cứu cũng giả thiết về thông tin kênh nghe lén mà bên phát có được là hoàn hảo hoặc chính xác một phần  \cite{kwon2012secure,wu2011information,he2014secrecy}. Giả thiết này thường khó đạt được trong thực tế, vì kẻ tấn công thường chỉ nghe lén và không có phản hồi tín hiệu tới bên phát. Thứ ba, mặc dù việc sử dụng các bộ phát lại giúp cải thiện chất lượng truyền tin và đảm bảo an toàn \cite{bouabdellah2018secrecy,quach2017secrecy}, tính tin cậy và đồng bộ với bên thứ ba này cũng là một vấn đề cần quan tâm. Do đó, bài toán thiết kế an toàn bảo mật trong mạng vô tuyến nhận thức cần một mô hình gần với thực tế, giảm thiểu những hạn chế nêu trên.

\section{Mục tiêu và định hướng giải pháp}

Với mục tiêu phát triển một giải pháp truyền tin an toàn cho mạng vô tuyến nhận thức, mô hình truyền tin và bài toán thiết kế trong đồ án này được xem xét trong các điều kiện gần với thực tế khi: (i) tính an toàn của PU và SU đều được cân nhắc khi thiết kế chiến lược truyền tin, (ii) trong điều kiện thông tin kênh truyền về kẻ nghe lén là hạn chế và các bên chỉ ước lượng được khoảng cách từ kẻ nghe lén tới các bên phát, (iii) chỉ sử dụng tính chất hợp tác giữa PU và SU có sẵn trong mạng để đảm bảo an toàn cho cả hai bên, thay vì phụ thuộc vào một bên thứ ba như các bộ phát lại. Mô hình đề xuất này giúp tránh được một số hạn chế trong các nghiên cứu trước đây, đồng thời cũng khai thác sự hợp tác để giải quyết bài toán thiết kế ban đầu.

Ngoài ra, vấn đề lợi ích của PU trong việc chia sẻ tài nguyên tần số cho SU cũng được xem xét, khi mà PU có thể lựa chọn một trong nhiều SU đem lại hiệu quả an toàn cao nhất cho mình. Khi đó, bài toán truyền tin an toàn ban đầu trở thành bài toán cạnh tranh giữa các SU, ở đó, các SU cần cân nhắc giữa cơ hội được truyền tin và chất lượng truyền tin an toàn trong việc thiết kế chiến lược truyền khi đề xuất hợp tác tới PU.

\section{Đóng góp của đồ án}

Đồ án này có các đóng góp chính như sau:

\begin{itemize}
\item Đề xuất chiến lược xác định công suất phát cho bên sơ cấp và thứ cấp của mạng vô tuyến nhận thức, đảm bảo truyền tin tin cậy và an toàn cho hai bên.
\item Khi xem xét tính cạnh tranh giữa các bên thứ cấp, khái niệm về "an toàn một phần" (partial secrecy) cũng được ứng dụng để các bên thứ cấp cân nhắc giữa mức độ an toàn và cơ hội được truyền tin cùng với bên sơ cấp.
\item Các độ đo hiệu năng như lượng tin bị nghe lén và xác suất truyền tin tin cậy được biểu diễn cụ thể cho mô hình hệ thống đã nêu.
\item Đồ án cũng cho thấy một hướng khai thác sự đa dạng về vị trí của các bên thứ cấp trong mạng để giúp cải thiện mức độ an toàn cho bên sơ cấp.
\end{itemize}

\section{Bố cục đồ án}

Phần còn lại của báo cáo đồ án tốt nghiệp này được tổ chức như sau:

Nội dung Chương \ref{chap2} tập trung trình bày nền tảng lý thuyết và các kết quả nghiên cứu liên quan tới các công nghệ mà đồ án sử dụng. Mục tiêu của chương này là làm rõ ý tưởng cốt lõi của công nghệ bảo mật lớp vật lý và giới thiệu một số kỹ thuật giúp cải thiện mức độ an toàn. Trên cơ sở đó, chương này cho thấy một hướng tiếp cận giúp giải quyết vấn đề an toàn bảo mật trong mạng vô tuyến nhận thức. Để lượng hóa mức độ hiệu quả của giải pháp, chương này cũng giới thiệu một số độ đo đánh giá mức độ an toàn thường được sử dụng trong các nghiên cứu về bảo mật lớp vật lý. Cuối chương trình bày về lý thuyết tối ưu, một công cụ quan trọng giúp giải quyết các bài toán thiết kế.

Chương \ref{chap3} trình bày kỹ hơn về mô hình hệ thống và mô hình truyền tin cho bài toán cần giải quyết. Với việc xác định rõ các đối tượng trong hệ thống cũng như đặc điểm kênh truyền, chương này lựa chọn các độ đo hiệu năng cho bài toán thiết kế. Cuối chương là phát biểu về bài toán tối ưu, mô hình cho bài toán thiết kế mà đồ án hướng tới giải quyết.

Với mô hình và bài toán xây dựng trong Chương \ref{chap3}, Chương \ref{chap4} sẽ trình bày các phân tích và hướng giải quyết bài toán. Quá trình phân tích giúp làm rõ mối quan hệ giữa các đại lượng trong bài toán, tương ứng với các lợi ích khác nhau của các đối tượng trong hệ thống. Trên cơ sở phân tích đó, chương này trình bày các hướng giải quyết bài toán ban đầu. Một mô hình mở rộng cho bài toán cũng được đề xuất trong chương này.

Từ kết quả thu được trong Chương \ref{chap4}, phương pháp thử nghiệm và kết quả triển khai với số liệu sẽ được thể hiện trong Chương \ref{chap5}. Các kết quả thử nghiệm góp phần chứng minh tính hiệu quả và nêu lên những hạn chế của giải pháp đề xuất.

Chương \ref{chap6} là phần tổng kết các kết quả đạt được của đồ án này. Trên cơ sở các kết quả phân tích và thử nghiệm, chương này cũng nêu ra một số hướng phát triển trong tương lai, một phần giúp vượt qua các hạn chế của giải pháp đề xuất, một phần giúp mô hình hệ thống gần với thực tế.

\end{document}