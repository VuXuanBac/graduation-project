\documentclass[../main.tex]{subfiles}
\begin{document}\label{appendixB}

Phụ lục \ref{appendixA} cho thấy các dung lượng kênh $C^{(E)}$, $C^{(F)}$ ứng với các bên nghe lén $\text{EAVP}$ và $\text{EAVS}$ có thể biểu diễn theo biến ngẫu nhiên $U$ định nghĩa trong \eqref{lemmaU:U}. Mặt khác, mức độ không rõ ràng của các kẻ nghe lén, $\Delta^{(P)}$ và $\Delta^{(S)}$ tương ứng, được biểu diễn theo các dung lượng kênh này. Từ công thức biểu diễn \eqref{deltap} và \eqref{deltas}, thấy rằng biến ngẫu nhiên $Z = h_{m,n}\left(U\right)$ định nghĩa trong \eqref{lemmaZ:Z} có thể đại diện cho các đại lượng $\Delta^{(P)}$ và $\Delta^{(S)}$. Do đó, phần này thực hiện khảo sát biến ngẫu nhiên $Z$.

\section{Hàm phân phối tích lũy}

Dễ thấy $Z=h_{m,n}\left(U\right)$ định nghĩa trong \eqref{lemmaZ:Z} chỉ nhận giá trị trên đoạn $\left[0, 1\right]$, nên $F_Z(z) = 0$ với $z \leq 0$ và $F_Z(z) = 1$ với $z > 1$. Xét $z \in \left(0, 1\right]$, ta có:
\begin{equation*}
\begin{aligned}
    F_Z(z) 
    &= \mathbb{P}\left(h\left(U\right) \leq z\right)
    \\ &= \mathbb{P}\left(2^m-1 \leq U\right) + \mathbb{P}\left(2^{m-n} - 1 < U < 2^m - 1\right)\\ & \qquad.\mathbb{P}\left(\,\frac{m - \log_2\left({1 + U}\right)}{n} \leq z \mid 2^{m-n} - 1 < U < 2^m - 1\,\right)
    \\ &= \mathbb{P}\left(2^m-1 \leq U\right) + \mathbb{P}\left(2^{m-nz} - 1 \leq U < 2^m - 1\right)
    \\ &= 1 - F_U\left(2^m-1\right) + F_U\left(2^m - 1\right) - F_U\left(2^{m-nz} - 1\right)
    \\ &= 1 - F_U\left(2^{m-nz} - 1\right),
\end{aligned}
\end{equation*}
với $F_U\left(u\right)$ là hàm phân phối tích lũy của $U$.

Tóm lại, hàm phân phối tích lũy của biến ngẫu nhiên $Z = h_{m,n}\left(U\right)$ có công thức:
\begin{equation}
F_Z\left(z\right) = 
\begin{cases}
1,& \text{nếu}\quad z > 1\\
1-F_U\left(2^{m-nz}-1\right),& \text{nếu}\quad 0 < z \leq 1\\
0 ,& \text{nếu}\quad z \leq 0.
\end{cases}
\end{equation}

\section{Giá trị kỳ vọng}

Do $Z$ chỉ nhận giá trị trên đoạn $\left[0,1\right]$, nên theo \cite[định lý 5.3.8, trang 230]{blitzstein2019introduction}, kỳ vọng của $Z$ được xác định là:
\begin{alignb}\label{proofB:EZ1}
    \mathbb{E}\left\{Z\right\} 
    &= \int_{0}^{\infty} \mathbb{P}\left(Z > z\right) \, dz \\
    &= \int_{0}^{1} \Bigl( 1 - F_Z\left(z\right) \Bigr) \, dz \\
    &= \int_{0}^{1} F_U\left(2^{m-nz}-1\right) \, dz.
\end{alignb}
Đặt $u = 2^{m-nz}-1$, ta có 
$du = -n\ln{2}\left(u+1\right)dz$, khi đó \eqref{proofB:EZ1} trở thành:
\begin{equation}\label{proofB:EZ}
    \mathbb{E}\left\{Z\right\} 
    = \frac{1}{n\ln{2}}\int_{2^{m-n}-1}^{2^{m}-1} 
    \frac{F_U\left(u\right)}{u+1} \, du.
\end{equation}

Các phần tiếp theo xác định kỳ vọng của $Z$ ứng với biến ngẫu nhiên $U$ có phân phối khác nhau.

\subsection{Phụ thuộc vào biến ngẫu nhiên có phân phối mũ}

Khi $U$ là biến ngẫu nhiên có phân phối mũ với kỳ vọng $\Omega_U$, ta có:
\begin{equation}\label{proofB:Uexp}
    F_U\left(u\right) = 1-\exp\left(-\frac{u}{\Omega_U}\right).
\end{equation}
Thay \eqref{proofB:Uexp} vào \eqref{proofB:EZ}, ta được:
\begin{alignb}\label{proofB:EZ:Uexp}
    \mathbb{E}\left\{Z\right\} 
    &= \frac{1}{n\ln{2}}\int_{2^{m-n}-1}^{2^{m}-1} \frac{1-\exp\left(-\frac{u}{\Omega_U}\right)}{u+1} \, du \\
    &= \frac{1}{n\ln{2}}\int_{2^{m-n}-1}^{2^{m}-1} \frac{1-\exp\left(-\frac{u}{\Omega_U}\right)}{u+1} \, du \\
    &= \frac{1}{n\ln{2}}\left[
        \ln\left(u+1\right)-\exp\left(\frac{1}{\Omega_U}\right)
                 \text{Ei}\left(-\frac{u+1}{\Omega_U}\right)
    \right]_{u \ = \ 2^{m-n}-1}^{u \ = \ 2^{m}-1} \\
    &= 1 - \frac{1}{n\ln{2}}\exp\left(\frac{1}{\Omega_U}\right)
    \left[
        \text{Ei}\left(-\frac{2^{m}}{\Omega_U}\right)
      - \text{Ei}\left(-\frac{2^{m-n}}{\Omega_U}\right)
    \right].
\end{alignb}

\subsection{Phụ thuộc vào hai biến ngẫu nhiên độc lập có phân phối mũ}

Khi $U$ là biến ngẫu nhiên được định nghĩa trong \eqref{lemmaU:U}, ta có:
\begin{equation}\label{proofB:EZ:U2}
    F_U\left(u\right) = 1-\frac{1}{\frac{b\Omega_Y}{a\Omega_X}u + 1}\exp\left(-\frac{uc}{a\Omega_X}\right).
\end{equation}
Để đơn giản, đặt $k = \frac{c}{a\Omega_X}$ và $l = \frac{b\Omega_Y}{a\Omega_X}$, khi đó, \eqref{proofB:EZ:U2} thành:
\begin{equation}\label{proofB:EZ:U3}
    F_U\left(u\right) = 1-\frac{\exp\left(-ku\right)}{lu + 1}.
\end{equation}
Thay \eqref{proofB:EZ:U3} vào \eqref{proofB:EZ}, ta được:
\allowdisplaybreaks
\begin{alignb}\label{proofB:EZ:Uexp:2}
    \mathbb{E}\left\{Z\right\} 
    &= \frac{1}{n\ln{2}}
    \int_{2^{m-n}-1}^{2^{m}-1} \left[
        \frac{1}{u+1} - 
        \frac{\exp\left(-ku\right)}{\left(lu + 1\right)\left(u+1\right)} 
    \right]\, du.
\end{alignb}

Trong trường hợp $l \ne 1$, ta có thể biến đổi \eqref{proofB:EZ:Uexp:2} thành:
\begin{alignb}\label{proofB:EZ:Uexp:21}
    \mathbb{E}\left\{Z\right\} 
    &= \frac{1}{n\ln{2}}
    \int_{2^{m-n}-1}^{2^{m}-1} \left[
        \frac{1}{u+1} 
        - \frac{1}{l-1}\left(
            \frac{\exp\left(-ku\right)}{u+1/l} 
            - \frac{\exp\left(-ku\right)}{u+1} 
        \right)
    \right]\, du \\
    &= \frac{1}{n\ln{2}}\left[
        \ln\left(u+1\right)
        - \frac{\exp\left(k/l\right)}{l-1}\text{Ei}\left(-ku - k/l\right) 
        + \frac{\exp\left(k\right)}{l-1}\text{Ei}\left(-ku-k\right)  
    \right]_{u \ = \ 2^{m-n}-1}^{u \ = \ 2^{m}-1} \\
    &= 1 
       - \frac{\exp\left(k/l\right)}{n\left(l-1\right)\ln{2}}
        \left[
            \text{Ei}\left(-k{2^{m}}+k-\frac{k}{l}\right)
          - \text{Ei}\left(-k{2^{m-n}}+k-\frac{k}{l}\right)
        \right] + \\
    & \qquad + \frac{\exp\left(k\right)}{n\left(l-1\right)\ln{2}}
        \Bigl[
            \text{Ei}\left(-k{2^{m}}\right)
          - \text{Ei}\left(-k{2^{m-n}}\right)
        \Bigr].
\end{alignb}

Trong trường hợp $l=1$, dễ dàng chứng minh kết quả sau:
\begin{alignb}\label{proofB:EZ:Uexp:22}
    \mathbb{E}\left\{Z\right\} 
    &= \frac{1}{n\ln{2}}\int_{2^{m-n}-1}^{2^{m}-1} 
        \left[
            \frac{1}{u+1} - \frac{\exp\left(-ku\right)}{\left(u+1\right)^2} 
        \right]\, du \\
    &= 1 + \frac{1}{n\ln{2}}
        \left[ 
            \frac{\exp\left(-k2^m+k\right)}{2^m} -
            \frac{\exp\left(-k2^{m-n}+k\right)}{2^{m-n}}
        \right] +  \\
       & \qquad + \frac{k\exp\left(k\right)}{n\ln{2}}
       \Bigl[
            \text{Ei}\left(-k{2^{m}}\right) - 
            \text{Ei}\left(-k{2^{m-n}}\right)
       \Bigr].
\end{alignb}

\end{document}