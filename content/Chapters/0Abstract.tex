\documentclass[../main.tex]{subfiles}
\begin{document}

\begin{center}
    \Large{\textbf{TÓM TẮT NỘI DUNG ĐỒ ÁN}}\\
\end{center}
\vspace{1cm}

Bảo mật lớp vật lý là một giải pháp tiềm năng giúp tăng cường an toàn thông tin trong truyền thông không dây. Cùng với các kỹ thuật mã hóa và xử lý tín hiệu, hợp tác cũng là một hướng tiếp cận giúp cải thiện mức độ an toàn. Một trong những mô hình mạng hợp tác tiềm năng cho công nghệ mạng di động thế hệ thứ năm (fifth generation - 5G) là mạng vô tuyến nhận thức, được giới thiệu với mục đích nâng cao hiệu quả sử dụng phổ. Trong mạng, bên sơ cấp có quyền sử dụng một phần tài nguyên tần số và sẵn sàng chia sẻ để các bên thứ cấp có thể khai thác sử dụng. Để truyền tin trên phần tài nguyên tần số của bên sơ cấp, bên thứ cấp có thể triển khai chiến lược truy cập phổ đồng thời, ở đó hai bên đồng thời truyền tin trong cùng khoảng thời gian và trên cùng dải tần số. Chiến lược này mặc dù giúp tận dụng hiệu quả tài nguyên tần số, song can nhiễu do tín hiệu từ bên thứ cấp lại làm giảm chất lượng tín hiệu nhận tại bên sơ cấp. Thế nhưng, khi xem xét thêm yêu cầu truyền tin an toàn, can nhiễu này có thể được tận dụng hiệu quả giúp bên sơ cấp cải thiện chất lượng truyền tin an toàn. Do đó, đồ án này đề xuất một giải pháp giúp lựa chọn công suất phát cho bên sơ cấp và thứ cấp của mạng vô tuyến nhận thức trong điều kiện có kẻ nghe lén ở cả hai bên, đảm bảo chất lượng truyền tin tin cậy và an toàn, ngay cả khi tốc độ truyền là cố định và các bên phát chỉ có thông tin thống kê về kênh truyền. Đồng thời, bằng việc mở rộng mô hình với nhiều bên thứ cấp, kết quả cho thấy bên sơ cấp có thể được lợi trong chiến lược hợp tác này. Với mô hình mở rộng, các bên thứ cấp cạnh tranh với nhau để được bên sơ cấp lựa chọn cùng hợp tác. Lúc này, khi là bên đề xuất chiến lược truyền, các bên thứ cấp phải cân nhắc lợi ích của cả hai bên để cân bằng giữa hai mục tiêu, đó là được truyền tin và chất lượng truyền tin an toàn.

\end{document}