\documentclass[../main.tex]{subfiles}
\begin{document}\label{chap6}

\section{Kết luận}

Trong quá trình hoàn thành đồ án tốt nghiệp, vấn đề an toàn bảo mật trong mạng vô tuyến nhận thức đã được đề cập, phân tích, đề xuất hướng giải quyết cũng như thử nghiệm đánh giá kết quả. Kết quả phân tích cho thấy vấn đề cần giải quyết là một vấn đề phức tạp khi hệ thống có nhiều người dùng khác nhau, với các lợi ích khác nhau, và thậm chí các lợi ích đó cũng xung đột với nhau. Để giúp giải quyết vấn đề này, đồ án đề xuất một chiến lược hợp tác giữa bên sơ cấp và bên thứ cấp thông qua một bài toán tối ưu đa mục tiêu. Trong đó, bên sơ cấp có quyền ưu tiên hơn trong vấn đề thiết kế, trong khi, bên thứ cấp có thể chủ động tùy chỉnh các mục tiêu và tham số thiết kế để cần cân bằng giữa hai lợi ích: được truyền tin và truyền tin an toàn. Trên cơ sở đó, bài toán ban đầu được phát triển và mở rộng, tương ứng phản ánh các lợi ích khác nhau của người dùng khác nhau. Kết quả thử nghiệm với các đề xuất đã chứng minh cho những phân tích trong đồ án. Tuy nhiên, do mô hình đề xuất còn đơn giản, các công nghệ và kỹ thuật xử lý tín hiệu chưa được khai thác sử dụng, nên kết quả thu được còn hạn chế và bị ảnh hưởng nhiều bởi điều kiện truyền tin. Song, với kết quả đạt được, đồ án cũng cho thấy tiềm năng của giải pháp đề xuất. Đồ án hy vọng rằng những đóng góp này sẽ được cộng đồng đón nhận, tiếp tục được khai thác và phát triển, góp phần hình thành một giải pháp cho vấn đề an toàn bảo mật trong mạng vô tuyến nhận thức.

\section{Hướng phát triển trong tương lai}

Kết quả của đồ án này còn nhiều hạn chế và bị chi phối nhiều bởi các điều kiện của môi trường truyền tin. Để vượt qua những hạn chế đó, hướng phát triển trong tương lai của đồ án tập trung vào việc khai thác những kỹ thuật xử lý tín hiệu cũng như các công nghệ hỗ trợ. Kỹ thuật định hướng búp sóng hay sinh nhiễu nhân tạo vốn là các kỹ thuật hiệu quả và được sử dụng nhiều trong các nghiên cứu về an toàn bảo mật lớp vật lý. Khi đó, mô hình hệ thống trong đồ án này có thể phát triển theo hướng sử dụng nhiều ăng-ten cho các bên nhằm định hướng tín hiệu tới người dùng hợp pháp hay gây nhiễu cho kẻ nghe lén. Bề mặt phản xạ thông minh (reconfigurable intelligent surfaces - RIS) cũng cho thấy là một công nghệ hiệu quả giúp hệ thống đạt được hiệu năng mong muốn ngay cả khi điều kiện kênh truyền không tốt. Cuối cùng, đồ án cũng mong muốn mở rộng mô hình hệ thống thành nhiều đối tượng cùng hợp tác, trong đó các bên thứ cấp có thể có cơ hội lựa chọn một trong nhiều bên sơ cấp để hợp tác truyền tin.

\end{document}